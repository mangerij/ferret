%%%%%%%%%%%%%%%%%%%%%%%%%%%%%%%%%%%%%%%%%
% Short Sectioned Assignment - UConn
% LaTeX Template
% Version 1.0 (5/5/12)
%
% This template has been downloaded from:
% http://www.LaTeXTemplates.com
%
% Original author:
% Frits Wenneker (http://www.howtotex.com)

% New Author after many edits:
% John Mangeri (Univ. Conn.)
%
% License:
% CC BY-NC-SA 3.0 (http://creativecommons.org/licenses/by-nc-sa/3.0/)
%
%%%%%%%%%%%%%%%%%%%%%%%%%%%%%%%%%%%%%%%%%

%----------------------------------------------------------------------------------------
%	PACKAGES AND OTHER DOCUMENT CONFIGURATIONS
%----------------------------------------------------------------------------------------

\documentclass[paper=a4, fontsize=14pt]{scrartcl} % A4 paper and 11pt font size

\usepackage[T1]{fontenc} % Use 8-bit encoding that has 256 glyphs
\usepackage{fourier} % Use the Adobe Utopia font for the document - comment this line to return to the LaTeX default
\usepackage[english]{babel} % English language/hyphenation
\usepackage{amsmath,amsfonts,amsthm} % Math packages
\usepackage{dsfont}
\usepackage{amssymb} %because we all want the therefore symbol
\usepackage{lipsum} % Used for inserting dummy 'Lorem ipsum' text into the template
\usepackage{xcolor}
\usepackage{cancel}
\usepackage[mathscr]{euscript}
\usepackage{subfigure}
\usepackage{hyperref}
\usepackage{tikz}
\usepackage{listings}
\usetikzlibrary{fit}
\tikzset{%
  highlight/.style={rectangle,rounded corners,fill=red!15,draw,
    fill opacity=0.5,thick,inner sep=0pt}
}
\newcommand{\tikzmark}[2]{\tikz[overlay,remember picture,
  baseline=(#1.base)] \node (#1) {#2};}
%
\newcommand{\Highlight}[1][submatrix]{%
    \tikz[overlay,remember picture]{
    \node[highlight,fit=(left.north west) (right.south east)] (#1) {};}
}

%------Define UConn Blue---%
\definecolor{uconnblue}{RGB}{15,25,56} % defined by http://webtools.uconn.edu/standards/colors.html
%                                                          % also http://help.business.uconn.edu/index.php?pg=kb.page&id=252 (graphics standards)    
\newcommand*{\boxedcolor}{uconnblue}
\makeatletter
\renewcommand{\boxed}[1]{\textcolor{\boxedcolor}{%
  \fbox{\normalcolor\m@th$\displaystyle#1$}}}
\makeatother
\usepackage{sectsty} % Allows customizing section commands
\allsectionsfont{\raggedright \normalfont\scshape} % Make all sections left, the default font and small caps
\usepackage{graphicx}
\usepackage{multirow}
\numberwithin{equation}{section} % Number equations within sections (i.e. 1.1, 1.2, 2.1, 2.2 instead of 1, 2, 3, 4)
\numberwithin{figure}{section} % Number figures within sections (i.e. 1.1, 1.2, 2.1, 2.2 instead of 1, 2, 3, 4)
\numberwithin{table}{section} % Number tables within sections (i.e. 1.1, 1.2, 2.1, 2.2 instead of 1, 2, 3, 4)

\setlength\parindent{0pt} % Removes all indentation from paragraphs - comment this line for an assignment with lots of text

%-------------------------
% Margins
%------------------------
\usepackage[margin=1.4in]{geometry}
	\addtolength{\oddsidemargin}{-1.075in}
	\addtolength{\evensidemargin}{-1.075in}
	\addtolength{\textwidth}{1.95in}

	\addtolength{\topmargin}{-.775in}
	\addtolength{\textheight}{1.53in}
%-------------------------------
% Header with UConn Logo
%--------------------------------
\usepackage{fancyhdr}                           % Custom headers and footers
\pagestyle{fancyplain}                            % Makes all pages in the document conform to the custom headers and footers

\renewcommand{\headrulewidth}{0pt} % Remove header underlines
\renewcommand{\footrulewidth}{0pt} % Remove footer underlines
\setlength{\headheight}{13.6pt} % Customize the height of the header
%\setlength{\headwidth}{51.6pt}  % Customize the width of the header
\pagestyle{fancy}
\fancyhf{} % clear all fields

\renewcommand\footrule{\begin{minipage}{1\textwidth}
\hrule width \hsize height 2pt \kern 1mm \hrule width \hsize   
\end{minipage}\par}%

\renewcommand\headrule{
\begin{minipage}{1\textwidth}
\hrule width \hsize \kern 1mm \hrule width \hsize height 2pt 
\end{minipage}}%

\fancyhead[RO]{%
  \scshape
  \begin{tabular}[b]{@{}r@{}}
  %insert assignment title\\ % and class here
 \textcolor{uconnblue}{\textsf{\textsc{Ferret} \date{\normalsize\today}}} \\ \textcolor{uconnblue}{\textsf{Tutorial problems}}
  \end{tabular}}
\fancyhead[LO]{%
  \scshape
  \begin{tabular}[b]{@{}l@{}}
 %\includegraphics[height=0.53in]{uconnlogo.jpg}\,\,\,\,\,\,\,\,\,\,\,\,\,\,\,\,\,\,\\   \textcolor{uconnblue}{J. Mangeri}
%{\textbf{FOR INTERNAL USE ONLY}}
  \end{tabular}}

\fancyfoot[R]{\thepage}




%----------------------------------------------------------------------------------------
%	TITLE SECTION
%----------------------------------------------------------------------------------------

\newcommand{\horrule}[1]{\rule{\linewidth}{#1}} % Create horizontal rule command with 1 argument of height

\title{	
\normalfont \normalsize 
\textsc{} \\ [25pt]                                    % Your university, school and/or department name(s) (Currently Blank)
\horrule{0.5pt} \\[0.1cm] % Thin top horizontal rule
% The assignment title
\horrule{2pt} \\[0.7cm] % Thick bottom horizontal rule
}

\author{} % Your name (currently left Blank)

\date{\normalsize\today} % Today's date or a custom date

\begin{document}
%\tableofcontents
%\newpage


\newpage

\section*{}

\section*{Preface}

First and foremost, \textsc{Ferret} \emph{only} works in 3D, so some extra care must be taken to do simple test problems that have analytical solutions. In what follows, are a few such test problems that build up to the fully-implicit fully-coupled inhomogeneous problem used in the Nanoscale paper (and other on-going work):

\begin{itemize}
\item \texttt{problem1.i} -- gradient-search for minima of the Landau expansion
\item \texttt{problem2a.i} -- Poisson equation/TDLGD (constant field)
\item \texttt{problem2b.i} -- Poisson equation/TDLGD (depolarization fields and domains)
\item \texttt{problem3a.i} -- Poisson equation/TDLGD + polar-elastic coupled problems
\item \texttt{problem3b.i} -- TDLGD coupled with elasticity and applied eigenstrain (misfit)
\end{itemize}

This work is current, as of May 9th, 2017, with the \textsc{Ferret} master, so please update your branch with 

$$\texttt{cd /ferret}$$

\vspace*{-12pt}

$$\texttt{git pull}$$

\vspace*{-12pt}

$$\texttt{./configure}$$

\vspace*{-12pt}

$$\texttt{make}$$

Compilation should \textbf{not} take long as many files have not changed since the last time you updated it. Please let me (john.mangeri@uconn.edu) know if it does. These problems for the most part should only use one processor but the last one, I would suggest to use 4 if possible.


\newpage

\vspace*{-65pt}

\section*{Problem 1 (gradient-search for minima)}

\vspace*{-10pt}

\subsection*{Set up:}

For the following, consult the input file \texttt{problem1.i}. Consider the bulk free energy \emph{density}

$$f_\mathrm{bulk}(\textbf{P}) = - \alpha_1 \left(P_x^2 + P_y^2 + P_z^2 \right) + \alpha_{11} \left(P_x^4 + P_y^4 + P_z^4 \right) $$

where $\alpha_1 = 0.5$ and $\alpha_{11} = 0.25$. Clearly, if 

$$\frac{d f_\mathrm{bulk} }{d P_z} = 0 \,\, \mathrm{then} \,\, P_z = \pm \frac{\sqrt{\alpha_1}}{\sqrt{2} \sqrt{\alpha_{11}}} = \pm 1.$$ 

If we apply a small field, $E_z = + 0.1$,  in the z direction, then the minima should be located at $P_z =  -1.04668$. The total energy density for this problem is then

$$f_\mathrm{total} = f_\mathrm{bulk}(\textbf{P} ) + P_z E_z$$

To find the minima, we evolve the time-dependent Landau problem,

\begin{align}\tag{1}\label{eq:1}
\frac{\partial P_z}{\partial t} = - \gamma \frac{\delta F}{\delta P_z},
\end{align}

with $\gamma = 1$ and $\delta/\delta \textbf{P}$ denotes the variational derivative of the total energy w.r.t. a component of $\textbf{P}$.  In \textsc{MOOSE}, we initialize the PDE term-by-term. The linearized terms in Eqs. \ref{eq:1} is a \texttt{Kernel}:

\begin{itemize}
\item \texttt{type} = \texttt{BulkEnergyDerivativeSixth} $\equiv \delta F_\mathrm{bulk} /\delta P_z$
\item \texttt{type} = \texttt{ConstField} $\equiv \delta / \delta P_z \left(\int_\Omega d^3 \textbf{r}\, P_z E_z \right) = \int_\Omega  d^3 \textbf{r} \,E_z$
\end{itemize}

The total energy $F$ is related to the total free energy density by integration,

$$F = \int\limits_\Omega d^3 \textbf{r}  \left(  f_\mathrm{bulk}(\textbf{P} (\textbf{r}),\textbf{r}) + P_z (\textbf{r}) E_z \right).$$

This integral is discretized ($\Omega = \cup_h \Omega_h )$ in the finite element method and the PDE is solved for each element. Also, the variables $P_x$ and $P_y$ are trivial constants for this problem.

\vspace{-15pt}

\subsection*{Execution}

\vspace{-8pt}

If we choose a random paraelectric guess for the initial condition, then the components of $\textbf{P}$ are randomly (and negligibly) distributed about zero.  See \texttt{IntialCondition} in the \texttt{Variables} block.


By running the input file, equation \ref{eq:1} is evolved in time until the minima has been reached. The field acts as a driving force to roll down the energy surface. By time step 6, the simulation ends because change $F_\mathrm{bulk} (t_n)  - F_\mathrm{bulk} (t_{n-1})$ is sufficiently small between adjacent time steps. The found minima $P_z = - 1.045814$ which is in agreement with the analytical solution found by taking derivatives and solving for the minima. 

\newpage

\vspace*{-65pt}

\section*{Problem 2a: Poisson/TDLGD}

\vspace*{-10pt}

\subsection*{Set up:}

Now, let's make this slightly more complicated -- see \texttt{problem2a.i}. Let's introduce a new variable $\Phi = \Phi (\textbf{r})$ and the Poisson problem,

$$\nabla^2 \Phi = - \nabla \cdot \textbf{P}$$

which couples to the bound charge $\rho_b = - \nabla \cdot \textbf{P}$. The linearized terms in this equation give the following new \texttt{Kernels}:

\begin{itemize}
\item \texttt{type} = \texttt{Electrostatics} $\equiv \nabla^2 \Phi$
\item \texttt{type} = \texttt{PolarElectricEStrong} $\equiv \nabla \cdot \textbf{P}$ (three components)
\end{itemize}


At each time step, $\Phi$ and $\textbf{P}$ will be found such that this new equation is satisfied. The computational domain is 10x10x6 nm. Let's put a potential BC on the short sides such that $\partial \Phi / \partial z = +0.1$ (see \texttt{[BCs]}). Now let's choose the free energy density

$$f_\mathrm{total} = - \alpha_1 \left(P_x^2 + P_y^2 + P_z^2 \right) + \alpha_{11} \left(P_x^4 + P_y^4 + P_z^4 \right) + P_z \frac{\partial \Phi}{\partial z} $$

where $\alpha_1 = 0.5$ and $\alpha_{11} = 0.25$, and once again evolve (with the same initial conditions) 

\begin{align}\nonumber
\frac{\partial P_z}{\partial t} = - \gamma \frac{\delta F}{\delta P_z}.
\end{align}

We have, 

\vspace{-10pt}

\begin{itemize}
\item \texttt{type} = \texttt{PolarElectricPStrong} $\equiv \delta /\delta P_z \left( P_z \partial \Phi / \partial z \right)$
\end{itemize}

Once $\Phi$ is found (analytical solution is a linear function that depends on $z$), it will update the TDLGD equation accordingly, which will influence the polarization with a constant field as before. 

\subsection*{Execution}

The degrees of freedom of the problem have gone up because we have introduced a new variable. Once again, we find $P_z = -1.045814$ but with a new mechanism of generating a field: satisfying the Poisson problem instantly, getting a constant field and then using that field to influence the time-dependent paraelectric $\to$ ferroelectric energy minimization. Inspection of $\Phi$ in the mesh file shows that indeed the Laplace problem is satisfied. Note that the uniform state returns a Laplace problem ($\nabla \cdot \textbf{P} \approx 0$).

\newpage

\vspace*{-65pt}

\section*{Problem 2b: Poisson/TDLGD \hspace*{-4pt}(domains\hspace*{-4pt} + \hspace*{-4pt}depolarization)}

\vspace*{-10pt}

\subsection*{Set up:}

Now we will show how the two-way coupling works when you have domains and a free surface. Let's introduce a new energy density, the Lifshitz invariant terms. They are

\begin{align}\tag{2}
\hspace{-5pt}f_\mathrm{wall} &= \frac{1}{2} G_{11} \left[\left(\frac{\partial P_x}{\partial x} \right)^2 \hspace{-1pt}+\hspace{-1pt} \left(\frac{\partial P_y}{\partial y}\right)^2 \hspace{-1pt}+\hspace{-1pt}\left( \frac{\partial P_z}{\partial z} \right)^2\right] + G_{12} \left[\frac{\partial P_x}{\partial x}\frac{\partial P_y}{\partial y}\hspace{-2pt} +\hspace{-1pt} \frac{\partial P_y}{\partial y} \frac{\partial P_z}{\partial z}\hspace{-2pt} +\hspace{-1pt} \frac{\partial P_x}{\partial x} \frac{\partial P_z}{\partial z} \right] \\ \nonumber
&+ \hspace{-2pt} \frac{1}{2} G_{44}  \hspace{-2pt}\left[\left(\frac{\partial P_x}{\partial y} \hspace{-1pt}+ \hspace{-1pt} \frac{\partial P_y}{\partial x} \right)^2 \hspace{-6pt}+\hspace{-2pt} \left(\frac{\partial P_y}{\partial z} \hspace{-1pt}+ \hspace{-1pt}\frac{\partial P_z}{\partial y} \right)^2 \hspace{-6pt}+\hspace{-2pt} \left(\frac{\partial P_x}{\partial z} \hspace{-1pt}+\hspace{-1pt} \frac{\partial P_z}{\partial x} \right)^2  \hspace{-0.5pt}\right] \\ \nonumber
&+ \hspace{-2pt}\frac{1}{2} G'_{44} \hspace{-2pt}\left[\left(\frac{\partial P_x}{\partial y} \hspace{-1pt}- \hspace{-1pt}\frac{\partial P_y}{\partial x} \right)^2 \hspace{-6pt} + \hspace{-2pt} \left(\frac{\partial P_y}{\partial z}\hspace{-1pt} - \hspace{-1pt}\frac{\partial P_z}{\partial y} \right)^2 \hspace{-6pt}+  \hspace{-2pt}\left(\frac{\partial P_x}{\partial z}\hspace{-1pt} - \hspace{-1pt}\frac{\partial P_z}{\partial x} \right)^2 \hspace{-0.5pt} \right]. \\ \nonumber
\end{align}

Let's choose $G_{11} = 0.2768$, $G_{12} = 0$, $G_{44} = G_{44}' = 0.1384$ (the magnitude of these control the size of the domain wall). The \texttt{Kernel} is

\begin{itemize}
\item \texttt{type} = \texttt{WallEnergyDerivative} $\equiv \delta /\delta \textbf{P} \left( G_{ijkl} \partial P_i / \partial x_j \partial P_k /\partial x_l \right)$ for each component of $\textbf{P}$.
\end{itemize}

We now evolve

\begin{align}\nonumber
\frac{\partial \textbf{P}}{\partial t} = - \gamma \frac{\delta F}{\delta \textbf{P}} = - \gamma \frac{\delta}{\delta \textbf{P}} \int\limits_\Omega d^3 \textbf{r} \left[ f_\mathrm{bulk} +  f_\mathrm{wall} + \textbf{P} \cdot \nabla \Phi \right]. 
\end{align}

and solve $\nabla^2 \Phi = - \nabla \cdot \textbf{P}$ at \emph{each} time step. We choose natural boundary conditions (so-called open circuit). The Landau coefficients are the same as the previous two examples.

\subsection*{Execution}

This solve takes slightly longer (439.225 seconds on one processor (Intel i7 w/ Ubuntu OS) because the domain wall term acts a micro-force to drive the system into equilibrium while still satisfying the unfavorable depolarization action of the Poisson problem (or so I understand) -- this gives rise to the vortex-like pattern seen in the solution. Unfortunately, we don't have analytical solutions for this, but this is essentially the simplest "generation" of a domain structure. Next, we will add elastic fields and their influence on the problems.

\newpage

\vspace*{-65pt}

\section*{Problem 3a: Poisson/TDLGD \hspace*{-4pt}(elasticity)}

\vspace*{-10pt}

\subsection*{Set up:}

Now we aim to couple the problem to elasticity. We do this by considering a stress-free strain caused by the polarization field \cite{Li2001, Mangeri2017},

$$\varepsilon_{ij}^0 = Q_{ijkl} P_k P_l,$$

where $Q_{ijkl}$ is the electrostrictive coefficient tensor. One then writes the total elastic free energy density as

$$\tilde{f}_\mathrm{elastic} = \frac{1}{2} C_{ijkl} \left( \varepsilon_{ij} - \varepsilon_{ij}^0\right)\left( \varepsilon_{kl} - \varepsilon_{kl}^0 \right),$$

with $C_{ijkl}$ being the elastic stiffness tensor and $\varepsilon_{ij}$ being elastic strains as functions of a new variable $\textbf{u}$, the elastic displacement fields. We can write this as

$$\tilde{f}_\mathrm{elastic} \simeq f_\mathrm{elastic} + f_\mathrm{coupled} = \frac{1}{2} C_{ijkl} \varepsilon_{ij} \varepsilon_{kl} + \frac{1}{2} q_{ijkl} \varepsilon_{ij} P_k P_l$$

where $q_{ijkl} = 2 Q_{ijmn} C_{mnkl}$. We then include these free energy densities in the TDLGD expression

\begin{align}\nonumber
\frac{\partial \textbf{P}}{\partial t} =  - \gamma \frac{\delta}{\delta \textbf{P}} \int\limits_\Omega d^3 \textbf{r} \left[ f_\mathrm{bulk} +  f_\mathrm{wall} + \textbf{P} \cdot \nabla \Phi  + f_\mathrm{elastic} + f_\mathrm{coupled}\right]. 
\end{align}

Note that the second-to-last term does not explicitly depend on the polarization field, so it is effectively zeroed under the variational derivative. In order to have two-way coupling to the polarization field, we need to also consider mechanical equilibrium of the stress field, $\sigma_{ij}$ , which takes the form of the stress-divergence problem,

$$\frac{\partial \sigma_{ij} }{\partial x_j} = \frac{\partial}{\partial x_j} C_{ijkl} \left(\varepsilon_{ij} - \varepsilon_{ij}^0 \right) = 0,$$

which is solved at each time step in the TDLGD evolution. For this example, we consider only natural boundary conditions on all of the variables ($\textbf{P}, \textbf{u}, \Phi$). This corresponds to a stress-free boundary condition for the elastic field. For these new additions:

\vspace*{-5pt}

\begin{itemize}
\item \texttt{[./TensorMechanics]} $\equiv \partial /\partial x_j \left( C_{ijkl} \varepsilon_{kl}\right)$ for each $i$ (stress-divergence).
\item \texttt{type} = \texttt{FerroelectricCouplingX} $\equiv -\partial /\partial x_j \left( C_{ijkl} \varepsilon_{kl}^0\right)$ for each $i$.
\item \texttt{type} = \texttt{FerroelectricCouplingP} $\equiv \delta / \partial \textbf{P} \int_\Omega \left(\frac{1}{2} q_{ijkl} \varepsilon_{ij} P_k P_l \right)$

\texttt{[Materials]}:

\vspace*{-10pt}

\begin{itemize}

\item \texttt{type} = \texttt{ComputeElasticityTensor} $\equiv$ fills in $C_{ijkl}$

\item \texttt{type} = \texttt{ComputeElectrostrictiveTensor} $\equiv$ fills in $q_{ijkl}$

\item \texttt{type} = \texttt{ComputeSmallStrain} 
\item \texttt{type} = \texttt{ComputeLinearElasticStress} computes strain/ stress fields.

\end{itemize}

\end{itemize}

\subsection*{Execution}

The new degrees of freedom increase the total number by about a factor of 2. It is suggested to use 2-4 processors for this problem and the solve will take around an hour to two hours. A new time-stepper has been introduced to facilitate the stiff parts of the solve. This time-stepper will automatically decrease the size of the time step if the nonlinear iterations are too large. We see that by time step 200, the energy changes are approaching order $10^{-3}$ which means the simulation has converged to an energy minima. This time, a clear vortex has formed since the electrostriction has the effect to soften the domain walls that can arise in small particles \cite{Mangeri2017}.

\newpage

\vspace*{-65pt}

\section*{Problem 3b: TDLGD \hspace*{-4pt}(misfit eigenstrain)}

\vspace*{-10pt}

\subsection*{Set up:}

Typically, it is common to study ferroelectric materials on substrate where the substrate lattice constant is different from the ferroelectric. To introduce a constant lattice mismatch, we can introduce a new eigenstrain in addition to the ferroelectric eigenstrain (also known as the self-strain or stress-free strain).

$$\bar{f}_\mathrm{elastic} = \frac{1}{2} C_{ijkl} \left( \varepsilon_{ij} + \bar{\varepsilon}_{ij} - \varepsilon_{ij}^0\right)\left( \varepsilon_{kl} + \bar{\varepsilon}_{ij} - \varepsilon_{kl}^0 \right),$$

where $\bar{\varepsilon}_{ij}$ is a constant value and only nonzero for $\bar{\varepsilon}_{xx} =  \bar{\varepsilon}_{yy}  = \bar{\varepsilon}$. To show the effectiveness of this approach, let's consider the problem

\begin{align}\nonumber
\frac{\partial \textbf{P}}{\partial t} =  - \gamma \frac{\delta}{\delta \textbf{P}} \int\limits_\Omega d^3 \textbf{r} \left[ f_\mathrm{bulk} +  f_\mathrm{wall}  + f_\mathrm{elastic} + f_\mathrm{coupled}\right]. 
\end{align}

with $\partial \sigma_{ij} /\partial x_j = 0$ and natural boundary conditions. Note that the Poisson equation and relevant coupling (including the $\Phi$ variable) has been \emph{removed} to simplify this discussion and we keep the same material tensor coefficients ($\alpha_{ij}, C_{ijkl}, Q_{ijkl}$, ect.) as before. The variable $\bar{\varepsilon}$ is modulated by the \texttt{[GlobalParams]} $\texttt{prefactor}$ in the input file. Note, 

\begin{itemize}

\vspace{-15pt}

\item[] 

\texttt{[Materials]}:

\vspace*{-12pt}

\begin{itemize}

\item \texttt{type} = \texttt{ComputeEigenstrain} $\equiv$ fills in $\bar{\varepsilon}_{ij}$ with

\texttt{eigen}$\_$\texttt{base} = 'exx exy exz eyx eyy eyz ezx ezy ezz' 

defiing the appropriate nonzero components of the eigenstrain.

\end{itemize}

\end{itemize}

\vspace*{-30pt}

\subsection*{Execution:}

When \texttt{prefactor} = 0.02, this means the structure is in tension, and all of the polar vectors will point in-plane (and form $90^\circ$ domain walls). When \texttt{prefactor} = - 0.02, the structure is in compression and the polar vectors will point out-of-plane( and form $180^\circ$ ``stripe'' patterns). Remember to change the name of the output mesh when looking at differences of the prefactor term! Here, the solve terminates after \texttt{num}$\_$\texttt{steps} = 55, but removing this line will allow the system to reach the true energy minima given the paraelectric initial condition.

Note: head-to-head and tail-to-tail domain walls are allowed since the electrostatic penalties are zeroed in this example!



\newpage
\begin{thebibliography}{1}

\bibitem{Li2001}
Y. L. ~Li, S. Y. ~Hu, Z. K. ~Liu, and L-. Q. Chen,
%\newblock {Phase-field model of domain structures in ferroelectric thin films}.
\newblock \emph{Appl. Phys. Lett.} \textbf{78} 492--500 (2001).

\bibitem{Mangeri2017}
J.~Mangeri, Y.~Espinal, A.~Jokisaari, S. P. ~Alpay, S.~Nakhmanson, and O. Heinonen,
%\newblock {Topological phase transitions and intrinsic size effects in ferroelectric nanoparticles}.
\newblock {\em Nanoscale} \textbf{9}, 1616-1624 (2017).

\end{thebibliography}



\end{document}